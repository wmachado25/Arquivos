\documentclass[12pt,a4paper]{article}%
\usepackage{amsmath,amsfonts,amstext,amsthm}%
\usepackage{pstricks,pst-plot}%
\usepackage[dvips]{graphicx}%
\usepackage[brazil]{babel}%
\usepackage[utf8]{inputenc}%

\usepackage{enumerate}

\pagestyle{empty}%

\setlength{\topmargin}{-2.5cm} 
\setlength{\oddsidemargin}{-1cm} %%% top margin -3
\setlength{\evensidemargin}{1cm} 
\setlength{\textheight}{27cm}
\setlength{\textwidth}{18cm}

\newcommand{\comando}[1]{{\Large #1} \\ \noindent\rule {17.9cm}{0.05cm}}
\newcommand{\parte}[1]{\vspace{1cm}{\large #1} \\ \noindent\rule {17.9cm}{0.05cm}}


\begin{document}


\comando{Met\'odo de Euler}

Método de Euler Implícito

\parte{Sintaxe}

\texttt{[y,t] = EulerImp(f,y0,N,h)}\\

Entrada:

\begin{itemize}
\item \texttt{f} é a função que está associada a derivada de \texttt{y}
\item \texttt{$y_{0}$} é a condição inicial da EDO     
\item \texttt{N} é o número de iterações 
\item \texttt{h} é um número fixo associado ao metódo de Newton.
\end{itemize}

Saída:

\begin{itemize}
\item \texttt{[y,t]} é a solução aproximada da EDO ($\ref{a}$) após $N$ interações. 
\end{itemize}


\parte{Descrição}

Considere a seguinte EDO:
\begin{align}
\begin{cases}
y'(t) = f(y(t),t),\\
y(t_{0}) = y_{0},
\end{cases}
\label{a}
\end{align}
Inicialmente temos que determinar quem é a função $f(y(t),t)$, usando diferenças regressivas para estimar a derivada $y'(t)$ e isolando o termo $y(t+h)$  temos que
\[y(t+h)=y(t)+hf(y(t+h),t+h),\]
no algoritmo vamos representar essa equação por
\[ G=@(x) \ x-f(x,t)*h-yi.\]
Após identificarmos a função $f(y(t),t)$ e a condição inicial $y(:,1)=y(t_{0})$, o algoritmo vai substituir os valores na equação acima , obtendo dessa forma 
\[ G=@(x) \ x-f(x,t)*h-y_{1}, \ \ \mbox{onde} \ \ t=t_{0}+h\]
em seguida vamos usar o Metódo de Newton para calcular sua solução, que será dada por:
\[y(:,2)=newton(G,y_{1},N,h)}.\]
O processor será feita $N$ vezes, dessa forma o algoritmo vai determinar uma solução $y(:,N+1)$ aproximada para a EDO ($\ref{a}$) após $N$ iterações.





\parte{Exemplos}


%\begin{verbatim}
\textbf{Exemplo 1.} Considere o problema apresentado em ($\ref{a}$), e tome $f(y(t),t) = -2y(t)$, com condição inicial $y(0) = 1$.\\


\textbf{Exemplo 2.}  Considere o problema apresentado em ($\ref{a}$), e tome $f(y(t),t) = y(t)(1-y(t))$, com condição inicial $y(0) = 1/2$.\\




%\end{verbatim}

\end{document}
