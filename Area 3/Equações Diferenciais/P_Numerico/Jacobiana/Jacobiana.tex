\documentclass[12pt,a4paper]{article}%
\usepackage{amsmath,amsfonts,amstext,amsthm}%
\usepackage{pstricks,pst-plot}%
\usepackage[dvips]{graphicx}%
\usepackage[brazil]{babel}%
\usepackage[utf8]{inputenc}%

\usepackage{enumerate}

\pagestyle{empty}%

\setlength{\topmargin}{-2.5cm} 
\setlength{\oddsidemargin}{-1cm} %%% top margin -3
\setlength{\evensidemargin}{1cm} 
\setlength{\textheight}{27cm}
\setlength{\textwidth}{18cm}

\newcommand{\comando}[1]{{\Large #1} \\ \noindent\rule {17.9cm}{0.05cm}}
\newcommand{\parte}[1]{\vspace{1cm}{\large #1} \\ \noindent\rule {17.9cm}{0.05cm}}


\begin{document}


\comando{jacobiana}

Jacobiana de uma função

\parte{Sintaxe}

\texttt{[J]=jacobiana(f,x,h)}\\

Entradas:

\begin{itemize}
\item \texttt{f} é a função a qual deseja-se calcular a matriz jacobiana;
\item \text{x} é o ponto no qual a jacobiana será computada;
\item \text{h} é o tamanho do passo para calcular a próxima iteração.
%\item \texttt{h} é o extremo superior do intervalo   
%\item \texttt{N} é o número de iterações 
\end{itemize}

Saída:

\begin{itemize}
\item \texttt{[J]} é a aproximação da jacobiana de $f$ no ponto $x$. 
\end{itemize}


\parte{Descrição}

\texttt{[J]=jacobiana(f,x,h)} retorna a jacobiana
\begin{center}
$J=%
\begin{bmatrix}
\frac{\partial f_{1}}{\partial x_{1}} & \frac{\partial f_{1}}{\partial x_{2}}
& \cdots  & \frac{\partial f_{1}}{\partial x_{n}} \\ 
\frac{\partial f_{2}}{\partial x_{1}} & \frac{\partial f_{2}}{\partial x_{2}}
& \cdots  & \frac{\partial f_{2}}{\partial x_{n}} \\ 
\vdots  & \vdots  & \ddots  & \vdots  \\ 
\frac{\partial f_{m}}{\partial x_{1}} & \frac{\partial f_{m}}{\partial x_{2}}
& \cdots  & \frac{\partial f_{m}}{\partial x_{n}}%
\end{bmatrix}%
$
\end{center}
da função $f:\mathbb{R}^{n} \rightarrow \mathbb{R}^{m}$, calculada no ponto $x \in \mathbb{R}^{n}$, usando diferenças centrais para a aproximação da derivada. As entradas $f$ e $x$ devem ser declaradas como vetores coluna.

\parte{Exemplo}

\begin{verbatim}

f = @(x) [2*x(1)*x(2);x(2)*cos(x(1));sin(x(1))];
x=[pi/2;1];

J=jacobiana(f,x,0.01)

J =

    2.0000    3.1416
   -1.0000    0.0000
         0         0
\end{verbatim}

\end{document}
